\chapter{Model}

\subsection{Indices}
i: Employee\\
j: Shift\\

\subsection{Parameters}
Employee$_{min,j}$: Minimum employees working at time\\
Employee$_{max,j}$: Maximum employees working at time\\
Preference$_{i,j}$: Boolean Employee Shift Preference\\
WeightedPreference$_{i,j}$: Weighted Employee Shift Preference\\
Availability$_{i,j}$: Boolean employee availability at time\\
Shift$_{min,i}$: Minimum employee shifts per week\\
Shift$_{max,i}$: Maximum employee shifts per week

\subsection{Decision Variable}
1 if employee i is scheduled to work shift j; 0 otherwise.
$$x_{i,j} =
 \begin{cases}
   1 \\
   0 
   
 \end{cases}
 $$
%------------------------------------------------

\section{Constraints}


$$\begin{equation}\label{availability} \sum\limits_{i} Availability_{i,j} \ge Employee_{min,j} \end{equation}

$$

 For each i: \\
$$ \begin{equation}\label{daylimit}\sum\limits_{j = k}^{k+5} x_{i,( j\bmod{count(j)})} \le 2 \end{equation} $$

$$ \begin{equation}\label{employeeshifts} Shift_{min,i}\le \sum\limits_{j}x_{i,j} \le Shift_{max,i} \end{equation}$$


$$ \begin{equation}\label{preference} Preference_{i,j} =   Preference_{i,j} \cdot Availability_{i,j} \end{equation}$$


\section{Employee Shift Weighting}
The employee preference parameter is created to allow employees to designate shifts they prefer. The model then optimizes to give employees the shifts they desire. 

The preferences parameter is calculated such that:

$$\sum\limits_{j}Preference_{i,j} = \sum\limits_{j}Availability_{i,j}$$

Should an employee choose not to specify priority shifts, or if an employee chooses to prioritize all shifts, the preference matrix should thus be equally weighted, and equal to the availability parameter:

$$Preference_{i,j} = Availability_{i,j}$$

Based on the objective function, priority shifts are upweighted, and based on the specified constraints the non-priority shifts must be downweighted. The upweighting factor is designated as $\alpha$ and the downweighting factor is designated as $\beta$. 

Considering j = 4 with one specified priority shift:
$$ (1 + \alpha) + (1-\beta) + (1-\beta) + (1-\beta) = 4 $$

Thus, in this case: $$\alpha = 3\beta$$

Clearly, $\beta$ must always be less than one. However, we also consider that employees who specify only a single priority shift have more weight placed on that shift than someone who prioritizes all but one shift.  We specify that a prioritized shift may be no more than twice as weighted as a non-weighted shift, thus $\alpha<1$.

$$\alpha_{i} = \frac{\sum\limits_{j}Availability_{i,j} - \sum\limits_{j}Preference_{i,j}}{\sum\limits_{j}Availability_{i,j}}$$

Thus,

$$\beta_{i} = \alpha \frac{\sum\limits_{j}Preference_{i,j}}{\sum\limits_{j}Availability_{i,j} - \sum\limits_{j}Preference_{i,j}}$$

In addition, if $count(\beta_{i})=0$ or $count(\alpha_{i})=0$,
$$ \beta_{i} = \alpha_{i} = 0 $$ 

Thus, 
$$ \alpha_{i}*count(\alpha_{i}) +\beta_{i} * count(\beta_{i}) = \sum\limits_{j}Availability_{i,j}$$

The weighted preference matrix 

\begin{verbatim}
function [ weighted_preferences ] = weighted_shifts( availability, preference )
%WEIGHTED_SHIFTS Returns a weighted shift matrix 

% Get sizes for loops
[num_employees num_shifts ] = size( availability );
% Initialize matrix by copying availability 
weighted_preference = availability;

% Loop through each employee
for i = 1:num_employees

    % Count how many shifts they are available, and how many 
	% they prefer
	num_available = sum( availability(i,:) );
	num_preferred = sum( preference( i,: ) );

	if ( num_preferred == 0 ) || ( num_preferred == num_available )
		% If they do not prefer any shifts, or if they prefer every shift
		% The availability matrix weighting is correct (all 1s)
	else 
		% we upweights and downweights based on number of preferred shifts 

		% upweight calculation
		alpha = ( num_available - num_preferred ) / num_available;
		
		% downweight calculation
		beta = alpha * num_preferred / ( num_available - num_preferred );
		
		% Loop through shifts and set weight
		for j = 1:num_shifts
			if availability( i , j ) == 1 && preference( i , j ) == 1
				% upweight shift
				weighted_preference( i , j ) = 1 + alpha;
			elseif availability( i , j ) == 1 && preference( i , j ) == 0
				% downweight shift
				weighted_preference( i , j ) = 1 - beta;
			else
				% Already zero weight due to availability matrix
				% weighted_preference( i , j ) = 0;
			end
		end
	end
end


end

\end{verbatim}

\section{Objective Function}

$$ max~Z = \sum\limits_{i,j} x_{i,j} \cdot WeightedPreference_{i,j}$$

